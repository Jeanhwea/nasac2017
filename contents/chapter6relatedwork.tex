% !TEX root = ../paper.tex
%% chapter 6

\section{Related Works}
%
In 1993, Chang et al.~\cite{chang} first proposed the project management problem. The view of 
Chang is that software project management net(SPM-Net) can be used to 
schedule tasks and manage the resources of software development. 
In his article, Chang's project management problem is based on the simulative 
data, the reason leading to this is the real industrial data of software 
project management is very scanty. In 2002, Aguilar-Ruiz et al.~\cite{alba} made a 
further research on simulated software project data and proposed search-based 
method to solve project management problem. They proposed 
simulation arrangement for the work package to provide a plan that project 
managers can follow to arrange tasks. Like Chang’s work, Aguilar-Ruiz's work 
is also based on simulative data. In 2007, Alba and Chicano optimized the 
search algorithms for project management, and solve the project management 
problem using genetic algorithm. Their goal is using a search-based approach 
to reduce the final completion duration of a project. In 2009, Ren et al.~\cite{ren} first 
applied co-evolutionary algorithms to solve project management problem. 
Recently, Sarro et al.~\cite{sarro} proposed a a multi-objective decision support approach to help
balance project risks and duration against overtime, so that software
engineers can better plan overtime. 
At present, the search space of the work package based on the project 
management is more and more huge, the sequential algorithm is not so 
effective to solve such problems. Thus, finding a parallel algorithm has 
become a hot topic on research~\cite{pentico}.


In recent years, search-based project management problem has become an 
important branch of search-based software engineering, and has become a new 
field of research. At the same time, the number of papers related to search-
based project management problem is also rising, which makes many researchers 
willing to engage in search-based project management problem, so in turn 
provides a new platform for practice and innovation of the search-based 
project management problem~\cite{penta}.


The search-based algorithm is a compute-intensive method, which means the
computer's CPU will be used usually consumed a lot. Therefore, the traditional
sequential computing model cannot meet the requirement of increasing calculation
speed. In 2007, Alba and Chicano~\cite{pospichal} began using search-based methods to improve the
optimal solution of problems, and for the first time using a parallel code model
to test the efficiency of search-based methods. In recent
years, more search-based software engineering methods (such as simulated
annealing, climbing algorithms, evolutionary algorithms, tabu search, etc.) have
been used to solve project management problem, and these methods are usually
able to get good convergence solution on project management problem.

As the development of the algorithm, how to speed up the calculation of the algorithm
has become a hot spot in academic community. Nowsdays,  A general-purpose GPU(GPGPU) is uesd 
to improve the speed of calculation in software engineering. In 2012, Zhang et al.~\cite{zhang} apply GPGPU to
simulation for complex scenes and proposed a GPU-based parallel MOEAs~\cite{li}.
CUDA is a generic parallel computing architecture developed by NVIDIA, which can be used to improve the performance of GPGPU~\cite{langdon2} and also can be optimized by meta-heuristic algorithms~\cite{langdon1}.


%\textbf{THIS PARAGRAPH TALK TO TOOL, TO BE DELETED???}
At present, for project management problem, scholars usually establish the
mathematical model, and then start their research and optimization on the
pre-defined model. In general, a common project management tool which deal with
the mathematical models have not been implemented, so it is difficult to apply
the theory to the industrial project management process. In 2012, Stylianou~\cite{stylianou} and
Gerasimou first developed a tool for project management, which they named
IntelliSPM. The tool uses Matlab and Java programming language and 
supports staffing arrangement and resource allocation optimization. In
their work, Stylianou uses the fitness function to dynamic calculate the
dependencies between work packages, so the real project's dependencies may be
broken during the calculation. So in current software engineering practice, the
tool supporting project management is still lacking.

