%
% ---------- chapter 4 ----------
%

\section{Description and Design of Algorithm}

\subsection{Summary}
%
This paper aims to solve above-mentioned project management problem by using
evolutionary algorithm.

% this is a figure 6

There are two important steps in using the evolutionary algorithm. One is to
choose the representation of the problem's solution, the other is to define a
appropriate fitness function. In the process of project management, the work
packages in the project is allocated by simulation. See figure 6, Firstly, the
whole project is decomposed into several work packages by \emph{work breakdown
  structure}, and those work packages is arranged into a corresponding
\emph{work package sequence}. Secondly, according to the dependencies of the
work packages and the restriction of resource, each kinds of resources is
allocated to the corresponding work package by \emph{first-come-first-served}
(FCFS) algorithm. Finally, the project's overall duration is calculate by
simulation.


\subsection{The Representation of Solution}
%
For the above-mentioned problem, the representation of solution is defined as
follows.


This paper uses \emph{work package sequence} (hereinafter referred to as \emph{WPS}) to
represent a solution of project management problem. The representation of such a
solution is actually an priority arrangement sequence of the work packages in
the whole project, and the number of solutions for a project containing the $N$
work packages is $N!$, which is large enough to do random search.

%This is a figure 7
\begin{equation}
  T_1 \rightarrow T_5 \rightarrow T_6 \rightarrow T_4 \rightarrow T_8
  \rightarrow T_3 \rightarrow T_9 \rightarrow T_2 \rightarrow T_7
  \label{repr}
\end{equation}

For example, equation (\ref{repr}) is solution, the solution represents the work
package sequence in the priority of $T_1$, $T_5$, $T_6$, $T_4$, $T_8$, $T_3$,
$T_9$, $T_2$, $T_7$, while arranging the work packages. This representation is
beneficial to programming and intuitive.


\subsection{Fitness Function}
%
The above-mentioned representation of solution make each individual coded as
\emph{WPS}. The fitness value of \emph{WPS} is the project's overall duration,
which is calculated by simulating the assignment of work packages in the set of
package sequence, see equation (\ref{wps}).

For each \emph{WPS}, the algorithm to calculate corresponding project overall
duration is as follows. Firstly, according to first-come-first-served (FCFS)
rule, the front work packages in a \emph{WPS} has high priority to get
resources, need to allocate in the resource diagram and the workload of each
work package, it will execute allocation of resources for the priority of each
individual work package coding. After the allocation, each work package will
have the start time and end time. Then the maximum execution time for all work
packages is the overall duration of the project.

