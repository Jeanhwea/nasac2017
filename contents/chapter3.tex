% chapter3

\section{Project Management Problem}
%

\subsection{Background}
%

The core of project management issues is work package scheduling and
resource allocation.


FIGURE 2 is a typical software project management Gantt chart, which
illustrats work package arrangement of a real industry software
engineering.  Generally, all the important phase in software
engineering, such as project planning, requirements analysis,
programming and testing, and project deployment, are inseparable from
the allocation of work packages.  These work packages have a mutually
restrictive relationship.  Some work packages must be started when
other work packages are completed.  For instance, the analysis of
software requirements is often to take place after the project
planning is completed.  Some work packages can be done at the same
time, and there is no impact on each other, such as software
development and software testing can often be synchronized.  The goal
of project manager is using the shortest duration to complete each
work package for the software project.


\subsection{Definition}
%
The goal of this paper is to find an optimal solution or near optimal
optimal solution for the project management problem described as
follows:

\emph{
  For a software project of $N$ work packages and $M$ kinds of
  resources, there is a corresponding dependency between these work
  packages, and each resource can only be assigned to the specified
  work package.  It is necessary to arrange the order of the work
  package reasonably and make the overall construction period as short
  as possible under the condition of satisfying the work package
  dependency and the resource allocation restriction.
}


\subsection{Assumptions}
%
There threre assumptions for our project management problem.

\emph{Assumption I}: The software project plan can be decomposed into
a set of work packages containing $N$ elements $T = \{t_1, t_2, ...,
t_N \}$.  Each work package in set $T$ is an indivisible (ie, The work
package can not be split to other work package for the schedule), the
set of work packages have a pre-estimated workload, the workload of
the composition of the collection $E = \{e_1, e_2, ..., e_N \}$.  The
function $TE: T \rightarrow E$, for a given work package $t_i$,
produces an estimated workload $e_i = TE(t_i)$ for a work package
$t_i$, where $e_i$ is the estimated workload of $t_i$.


\emph{Assumption II}: The work packages in the software project plan
can be processed with $M$ kinds of resources, which constitute a
resource set $R = \{r_1, r_2, ..., r_M \}$, for each project's work
package set $T$ and the resource set $R$, There exsits a function
$TR: T \times R \rightarrow \{0, 1\}$, for the given work package $t_i$
and the resource $r_j$, $TR(t_i, t_j) = 1$ means the resource $r_j$ can
be allocated to the work package $t_i$, while $TR (t_i, r_j) = 0$ means
can not.


\emph{Assumption III}: All work packages in the work package of the
software project plan have dependencies.  These dependencies form a set
$Dep= \{t_i \rightarrow t_j \mid t_i, t_j \in T, t_j \text{ depends on } t_i\}$.
As the assumption of this paper, $t_i \rightarrow t_j$ means that
$t_j$ depends on $t_i$, that is the work package $t_j$ must be
arranged after the work package $t_i$, and satisfy the formula
$t_j.start \leq t_i.end$ (where $t.start$ and $t.end$ Respectively
indicate the start time and the end time of the work package $t$).
And assume that there is no direct or indirect "loop" dependency
between work packages.


\subsection{The Object of Problem}
%
The object of project management is to find a optimal work package sequence
under the three assumptions below.

The notation of work package sequence(WPS) is as follows:

\begin{equation}
  S = \{
  (t_{p_1}, r_{q_1}) ... \rightarrow (t_{p_j}, r_{q_j}) \rightarrow ... (t_{p_N}, r_{q_N})
  \mid t_{p_i} \in T, r_{q_j} \in R
  \}
\end{equation}

where every $(t_p, r_q)$ means resource $r_q$ is allocated to work package $t_p$.
The WPS needs to meet the following two restrictions:

\begin{enumerate}
\item $\nexists i < j, t_j \rightarrow t_i \in Dep$.
  (The WPS must satisfy dependencies)
\item $\nexists i, k, TR(t_i, r_k) = 0$.
  (All work packages must have at least one resource)
\end{enumerate}

For the work package arrangement sequence S, each work package $t$ is
given $t.start$ (the start time) and $t.end$ (the end time).  The
total cost duration function of the project is defined as following:

\begin{equation}
f(S) = max\{t.end \mid t \in T\}
\end{equation}

that is, the total duration f represents the maximum value of the end
time of all work packages in a work package arrangement sequence $S$,
which means when the last work package for the project is completed,
the entire project ends.

The objective of the project management problem is to find a work
package sequence $S$ to arrange the whole project under the condition
of satisfying the restriction of dependencies and resources, so that
the total duration is minimized.

