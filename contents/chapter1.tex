%% chapter 1

\section{Introduction}

\subsection{Background}
%
In our daily life, each of us are facing many work tasks.  How to properly
arrange those tasks is a hot research problem in academia. For example, in a car
production line, the production manager needs to split the project into several
tasks or work packages and then deliver the work package to each machine that
finishs the different tasks. Not only in the automobile production process, but
software project management is also the same. As we all know, in an actual
development process of software project, the project manager's responsibility is
to properly schedule the tasks of software project development, supervise the
programming work of the software engineer and arrange the development progress
of the software to ensure that the software can be delivered before the deadline
\cite{stellman}. Therefore, the project management problem is not only a very
fundamental problem in software engineering, but also a very difficult problem
in the actual work for the project manager.


Through a survey on software project management practice, we found that the
different task management designed by the project manager, such as the
arrangement order of the work packages in a project, or the resource allocation
in the same human resources team, will have a great impact on the project's
overall duration. Excellent project managers can shorten the overall duration by
arranging the order of tasks, or making full use of the team resources and
allocating human resources properly. However, at the beginning of a software
project, the project managers need to spend a lot of time on the discussion,
which find what kind of difficulties will be faced in the software project and
discuss the detail of resources allocation plan in software engineering phase,
such as requirements analysis, system design, system development, system testing
and etc. At present, an automatic method to properly arrange the entire project
work package is still lacking, which is not only the problem that project
manager faced in the actual daily work, but also the challenges that the
industry focus on. Therefore, it is necessary to design a algorithms that can
help the project manager manage the work packages.


In summary, software project management is an art of people management and task
scheduling in software engineering. It requires an overall understanding of the
lifecycle of software development, such as planning tasks, staff organization
and so on. The main purpose of this paper is to solve the problem of software
project management by the search-based approach, which is an important component
of search-based software engineering (SBSE) and it is also a future development
trend of software engineering project management under the big data age.

\subsection{Related Works}
%
In 1993, Chang first proposed the project management problem. The view of Chang
is that software project management net(SPM-Net) can be used to schedule tasks
and manage the resources of software development \cite{chang}. In his article,
Chang's project management problem is based on the simulated data, the reason
leading to this is the real industrial data of software project management is
very scanty. In 2002, Aguilar-Ruiz et al. made a further research on simulated
software project data and bring search-based method to solve project management
problem \cite{alba}. They proposed simulation arrangement for the work package
to provide a plan that project managers can follow to arrange each tasks. Like
Chang, Aguilar-Ruiz's work is also based on simulative data. In 2007, Alba and
Chicano optimized the search algorithms for project management, and solve the
project management problem using genetic algorithm. They goal is using a
search-based approach to reduce the final completion period of a project. In
2009, Ren first applied co-evolutionary algorithms to project management problem
\cite{ren}. At present, the search space of the work package based on the
project management is more and more huge, the sequential algorithm is not so
effective to solve such problems. Thus, Finding a parallel algorithm solution is
has become a hot topic on research \cite{pentico}.

In recent years, search-based project management has become an important branch
of search-based software engineering, which has become a new field of
research. At the same time, the number of papers related to search-based project
management issues is also rising, which makes many researchers willing to engage
in search-based project management issues, and also provide a new platform for
practice and innovation of for the search-based project management issues
\cite{penta}.

Because the search-based method is a computationally intensive method, the
computer's CPU computing resources are usually consumed very much. Therefore,
the traditional serial computing model cannot meet the requirement of
calculating the speed that satisfy the method based on the search. In 2007, Alba
and Chicano began using search-based methods to improve the optimal solution for
finding problems, and for the first time using a parallel code implementation to
test the efficiency of using search-based methods \cite{pospichal}.  In recent
years, more search-based software engineering methods (such as simulated
annealing, climbing algorithms, evolutionary algorithms, tabu search, etc.) have
been used to solve project management problems, and these methods are usually
able to get good convergence solution on project management issues.

At present, for the definition and solution of the project management problem,
scholars usually establish the relevant mathematical model in the process of
research, and then carry out research and optimization in their own model. In
general, these mathematical models are not developed into a common tool, so it
is difficult to apply to the actual production process of the project
manager. In 2012, Stylianou and Gerasimou first developed a tool for project
management, which they named IntelliSPM\cite{stylianou}. This is a tool that
uses Matlab and Java programming to support software project management, which
supports staffing arrangement and resource allocation optimization. Stylianou
uses the fitness evaluation function to control the dependencies between work
packages, so this often does not completely solve the dependency of the work
package. Stylianou did not disclose this tool.  At present, in engineering
practice, the project management is still lack of tool support. Facing the
growing needs of automate the scheduling of tools which are needed by software
project managers, developing a software project management tool that supports
project manager decision making has become a new goal in the field of search
project management.

