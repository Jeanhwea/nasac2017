%% chapter 1

\section{Introduction}

\subsection{Background}
% todo
In the production and life, each of us will come into contact with all kinds of
production tasks. for these large or small production tasks, how to properly
deploy the task is a hot issue to be solved urgently. For example, in a car
production line, the production manager needs to split a project that produces a
car into several sub-tasks or more elaborate work package and then deliver the
work package to each machine that carries different machining task. Not only in
the automobile production process so, software project management is also
according to the law. We know that in an actual development process of software
project, the project manager's responsibility is to properly schedule the tasks
of software project development, supervise the programming work of the software
engineer and arrange the development progress of the software to ensure that the
software can be delivered before the scheduled time \cite{stellman}. Therefore,
the project management problem is a very basic problem in software engineering,
but also is a most difficult problem in the actual work for the project manager.


Through the actual research of production practice, we can draw out that the
different task management schemes designed by the project manager, such as the
arrangement order of the different work packages in a project, or the different
resource allocation in the same human resources team and the rationality of the
members of the team, will have a great impact on the overall duration of a
project. Excellent project managers can shorten the overall duration of a
software project by allocating the order of execution of a task reasonably, or
make full use of the resources of the team and allocate human resources
rationally to maximize the overall profit for project. However, before the start
of a software project, company's project management staff, often need to spend a
lot of time to discuss what kind of difficulties faced in the software project
development process, and to discuss several times to make the allocation plan of
work packages such as Demand analysis, system design, system development, system
testing and etc.  At present, the industry is faced with a lack of automatic
method to properly arrange the entire project work package, which is not only
the problem that project manager faced in the actual work process, but also the
challenges what industry focus on solving. Therefore, it is necessary to design
a tool that can help the project manager manage the work packages.


Software Project Management is an art of people management and task scheduling
in software engineering. It requires an overall understanding of the lifecycle
of software development, such as planning tasks, staff organization and so
on. The main research point of this paper is to solve the problem of software
project management by using the search-based approach, which is an important
component of Search Based Software Engineering (SBSE) and it is also a future
development trend of software engineering project management under the big data
age.

\subsection{Related Works}
%
In 1993, Chang first proposed the project management problem. The view of Chang
is that software project management net(SPM-Net) can be used to schedule tasks
and manage the resources of software development \cite{chang}. In his article,
Chang's project management problem is based on the simulated data, the reason
leading to this is the real industrial data of software project management is
very scanty. In 2002, Aguilar-Ruiz et al. made a further research on simulated
software project data and bring search-based method to solve project management
problem \cite{alba}. They proposed simulation arrangement for the work package
to provide a plan that project managers can follow to arrange each tasks. Like
Chang, Aguilar-Ruiz's work is also based on simulative data. In 2007, Alba and
Chicano optimized the search algorithms for project management, and solve the
project management problem using genetic algorithm. They goal is using a
search-based approach to reduce the final completion period of a project. In
2009, Ren first applied co-evolutionary algorithms to project management problem
\cite{ren}. At present, the search space of the work package based on the
project management is more and more huge, the sequential algorithm is not so
effective to solve such problems. Thus, Finding a parallel algorithm solution is
has become a hot topic on research \cite{pentico}.

In recent years, search-based project management has become an important branch
of search-based software engineering, which has become a new field of
research. At the same time, the number of papers related to search-based project
management issues is also rising, which makes many researchers willing to engage
in search-based project management issues, and also provide a new platform for
practice and innovation of for the search-based project management issues
\cite{penta}.

Because the search-based method is a computationally intensive method, the
computer's CPU computing resources are usually consumed very much. Therefore,
the traditional serial computing model cannot meet the requirement of
calculating the speed that satisfy the method based on the search. In 2007, Alba
and Chicano began using search-based methods to improve the optimal solution for
finding problems, and for the first time using a parallel code implementation to
test the efficiency of using search-based methods \cite{pospichal}.  In recent
years, more search-based software engineering methods (such as simulated
annealing, climbing algorithms, evolutionary algorithms, tabu search, etc.) have
been used to solve project management problems, and these methods are usually
able to get good convergence solution on project management issues.

At present, for the definition and solution of the project management problem,
scholars usually establish the relevant mathematical model in the process of
research, and then carry out research and optimization in their own model. In
general, these mathematical models are not developed into a common tool, so it
is difficult to apply to the actual production process of the project
manager. In 2012, Stylianou and Gerasimou first developed a tool for project
management, which they named IntelliSPM\cite{stylianou}. This is a tool that
uses Matlab and Java programming to support software project management, which
supports staffing arrangement and resource allocation optimization. Stylianou
uses the fitness evaluation function to control the dependencies between work
packages, so this often does not completely solve the dependency of the work
package. Stylianou did not disclose this tool.  At present, in engineering
practice, the project management is still lack of tool support. Facing the
growing needs of automate the scheduling of tools which are needed by software
project managers, developing a software project management tool that supports
project manager decision making has become a new goal in the field of search
project management.

