%% chapter 01: 相关研究工作

\chapter*{Related Works}
%

In 1993, Chang, C proposed the problem of project management.
The view of Chang is that Software Project Management Net can
be used to schedule tasks and manage the resources of software
development. In his article, Chang's problem of project management is based 
on the industrial data which is simulated by computer, the reason leading to 
this is the real industrial data of software project management is very 
scanty. Then in 2002, Aguilar-Ruiz used simulative software project data to 
further research and put forward the project management method based on 
search. Their method put forward an approach of simulation to arrange the 
work package to provide a plan that project managers can follow to arrange 
the tasks [3]. Like Chang, Aguilar-Ruiz is also based on simulative data to 
evaluate their experimental results. In 2007, Alba and Chicano optimized the 
corresponding search algorithms for project management, and proposed the use 
of Genetic Algorithm to solve the problem of project management, using a 
search-based approach to reduce the final completion period of a project. In 
2009, Ren proposed the use of coevolutionary algorithms to solve project 
management issues [4]. At present, the search space of the work package based 
on the project management is more and more huge, the traditional serial 
solution to solve such problems has been unable to meet people's requirements 
for the speed of the calculation. Thus, the solution to the parallel method 
of such problems has become a hot topic of research [5].


In recent years, search-based project management has become an important 
branch of Search Based Software Engineering, which has become a new field of 
research. At the same time, the number of papers related to search-based 
project management issues is also rising, which makes many researchers 
willing to engage in search-based project management issues, and also provide 
a new platform for practice and innovation of for the search-based project 
management issues [6].


Because the search-based method is a computationally intensive method, the 
computer's CPU computing resources are usually consumed very much. Therefore, 
the traditional serial computing model cannot meet the requirement of 
calculating the speed that satisfy the method based on the search. In 2007, 
Alba and Chicano began using search-based methods to improve the optimal 
solution for finding problems, and for the first time using a parallel code 
implementation to test the efficiency of using search-based methods [7]. In 
recent years, more search-based software engineering methods (such as 
simulated annealing, climbing algorithms, evolutionary algorithms, tabu 
search, etc.) have been used to solve project management problems, and these 
methods are usually able to get good convergence solution on project 
management issues [8].


At present, for the definition and solution of the project management 
problem, scholars usually establish the relevant mathematical model in the 
process of research, and then carry out research and optimization in their 
own model. In general, these mathematical models are not developed into a 
common tool, so it is difficult to apply to the actual production process of 
the project manager. In 2012, Stylianou and Gerasimou first developed a tool 
for project management, which they named IntelliSPM [9]. This is a tool that 
uses Matlab and Java programming to support software project management, 
which supports staffing arrangement and resource allocation optimization. 
Stylianou uses the fitness evaluation function to control the dependencies 
between work packages, so this often does not completely solve the dependency 
of the work package. Stylianou did not disclose this tool. At present, in 
engineering practice, the project management is still lack of tool support. 
Facing the growing needs of automate the scheduling of tools which are needed 
by software project managers, developing a software project management tool 
that supports project manager decision making has become a new goal in the 
field of search project management.

