%% chapter 1

\section{Introduction}

\subsection{Background}
%
In our daily life, each of us are facing many work tasks.  How to properly 
arrange those tasks is a hot research problem in academia. For example, in an 
automobile production line, the production manager needs to split the project 
into several tasks or work packages and then deliver the work package to each 
machine that finishes the different tasks. Not only in the automobile 
production process, but software project management is also the same. As we 
all know, in an actual development process of software project, the project 
manager's responsibility is to properly schedule the tasks of software 
project development, supervise the programming work of the software engineer 
and arrange the whole development progress of the software to ensure that the 
software can be delivered before the deadline \cite{stellman}. Therefore, the 
project management problem is not only a very fundamental problem in software 
engineering, but also a very difficult problem in the actual work for the 
project manager.


Through a survey on the practice of software project management, we found 
that the different task management designed by the project manager, such as 
the arrangement order of the work packages in a project, or the different 
resource allocation in the same human resources team, will have a great 
impact on the project's overall duration. Excellent project managers can 
shorten the overall duration by arranging the order of tasks, making full use 
of the team’s resources or allocating human resources properly. However, at 
the beginning of a software project, the project managers need to spend a lot 
of time on the discussion that what kind of difficulties will be faced in the 
process of development and the detail of resources allocation in 
software engineering phase, such as requirements analysis, system design, 
system development, system testing and etc. At present, an automatic method 
to properly arrange the entire project work package is still lacking, which 
is not only the problem that project manager faced in the actual daily work, 
but also the challenges that the industry focus on. Therefore, it is 
necessary to design an algorithm that can help the project manager manage the 
work packages.


In summary, software project management is an art of staff management and task
scheduling in software engineering. It requires an overall understanding of the
lifecycle of software development, such as planning tasks, staff organization
and so on. The main purpose of this paper is to solve the problem of software
project management by the search-based approach, which is an important component
of search-based software engineering (SBSE) and also a future development
trend of software engineering project management under the big data age.


\subsection{Related Works}
%
In 1993, Chang first proposed the project management problem. The view of 
Chang is that software project management net(SPM-Net) can be used to 
schedule tasks and manage the resources of software development \cite{chang}. 
In his article, Chang's project management problem is based on the simulative 
data, the reason leading to this is the real industrial data of software 
project management is very scanty. In 2002, Aguilar-Ruiz et al. made a 
further research on simulated software project data and proposed search-based 
method to solve project management problem \cite{alba}. They proposed 
simulation arrangement for the work package to provide a plan that project 
managers can follow to arrange tasks. Like Chang’s work, Aguilar-Ruiz's work 
is also based on simulative data. In 2007, Alba and Chicano optimized the 
search algorithms for project management, and solve the project management 
problem using genetic algorithm. Their goal is using a search-based approach 
to reduce the final completion duration of a project. In 2009, Ren first 
applied co-evolutionary algorithms to solve project management problem 
\cite{ren}. At present, the search space of the work package based on the project 
management is more and more huge, the sequential algorithm is not so 
effective to solve such problems. Thus, finding a parallel algorithm has 
become a hot topic on research \cite{pentico}.


In recent years, search-based project management problem has become an 
important branch of search-based software engineering, and has become a new 
field of research. At the same time, the number of papers related to search-
based project management problem is also rising, which makes many researchers 
willing to engage in search-based project management problem, so in turn 
provides a new platform for practice and innovation of the search-based 
project management problem \cite{penta}.


Because the search-based algorithm is a compute-intensive method, which means 
the computer's CPU will be used usually consumed a lot. Therefore, the 
traditional sequential computing model cannot meet the requirement of 
increasing calculation speed. In 2007, Alba and Chicano began using search-based
 methods to improve the optimal solution of problems, and for 
the first time using a parallel code model to test the efficiency of search-based 
methods \cite{pospichal}. In recent years, more search-based software 
engineering methods (such as simulated annealing, climbing algorithms, 
evolutionary algorithms, tabu search, etc.) have been used to solve project 
management problem, and these methods are usually able to get good 
convergence solution on project management problem.


% todo 这段讲了工具,需要筛选
At present, for project management problem, scholars usually establish the
mathematical model, and then start their research and optimization on the
pre-defined model. In general, a common project management tool which deal with
the mathematical models have not been implemented, so it is difficult to apply
the theory to the industrial project management process. In 2012, Stylianou and
Gerasimou first developed a tool for project management, which they named
IntelliSPM \cite{stylianou}. The tool uses Matlab and Java programming language and 
supports staffing arrangement and resource allocation optimization. In
their work, Stylianou uses the fitness function to dynamic calculate the
dependencies between work packages, so the real project's dependencies may be
broken during the calculation. So in current software engineering practice, the
tool supporting project management is still lacking.

