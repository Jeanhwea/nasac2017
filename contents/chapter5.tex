\section{Evaluation}
%
For the validity assessment of the evolutionary algorithm proposed in this 
paper, this paper makes separate evaluation experiments for the efficiency of 
parallel evolutionary algorithm and the validity of evolutionary algorithm in 
project management problem. The purpose of the experiment is to answer the 
following two questions:
RQ1: Is the parallel evolution algorithm able to improve the efficiency of 
the schedule?
RQ2: Does the evolutionary algorithm effectively optimize project management 
issues?

\subsection{Experimental Data}
The experimental data evaluated in this paper are examples of real industrial 
project management. There are three software project plannings, which are 
named as A-Input, B-DBUpgrade and C-SmartPrice, where A-Input is the medium-
scale project planning of the simulation structure; B-DBUpgrade is the 
project plan of the Oracle’s database upgrade. The goal is to upgrade the 
Oracle database from the 9g version to the 10g version, which is Oracle's non-
public version of the project planning; C-SmartPrice is a medium-sized supply 
chain upgrade project planning, at the same time is a non-public Of the 
internal project planning\cite{ren}. Data related statistics show in Table 1.
%table 1

\subsection{Evaluation Experimeent for Algorithm Efficiency}
In order to answer RQ1, this section achieves the efficiency comparison test 
between the parallel evolution algorithm and serial evolution algorithm. The 
hardware environment of this experiment is as follows: the runtime 
environment that execute the serial evolution algorithm is the Intel i7 
series central processing unit (CPU). The runtime environment of the parallel 
evolution algorithm is the NVidia GeForce series graphics processing unit (GPU
). CPU and GPU detailed configuration comparison see Table 6.1. The software 
environment of this experiment is: Window 7 Ultimate Service Pack 1, 
Microsoft Visual Studio 2012 C ++ Compiler, CUDA 7.0.28 Runtime compilation 
environment. The specific hardware configuration is shown in Table 2.
%table 2


