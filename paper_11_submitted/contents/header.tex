% !TEX root = ../paper.tex
% header
%

\title{Parallel Evolutionary Algorithm in Scheduling Work Packages to Minimize Duration of Software Project Management}
%Solving Project Management Problem with Parallel Evolutionary Algorithm}
%
\titlerunning{Project Management}  % abbreviated title (for running head)
%                                     also used for the TOC unless
%                                     \toctitle is used
%
\author{ Jinghui Hu \and Xu Wang \and Jian Ren\thanks{Corresponding author: {J. Ren (renjian@buaa.edu.cn)}} \and Chao Liu}
%
\authorrunning{Hu, Wang, Ren, Liu} % abbreviated author list (for running head)
%
%%%% list of authors for the TOC (use if author list has to be modified)
\tocauthor{Jinghui Hu, Xu Wang, Jian Ren, Chao Liu}
%
\institute{\vspace{-0.8em}
State Key Laboratory of Software Development Environment,\\
School of Computer Science and Engineering, 
\\Beihang University, Beijing 100191, China
%\\
%\email{\{hujinghui, bhwangxu, renjian, liuchao\}@buaa.edu.cn}
}



\maketitle

\begin{abstract}
\vspace{-6mm}
Software project management problem mainly includes resources allocation and work packages scheduling.
This paper presents an approach to Search Based Software Project Management based on parallel implementation of evolutionary algorithm on GPU. 
We redesigned evolutionary algorithm to cater for the purpose of parallel programming.
Our approach aims to parallelize the genetic operators including: crossover, mutation and evaluation in the evolution process to achieve faster execution. % time. for the optimization process.
To evaluate our approach, we conducted a ``proof of concept'' empirical study, using data from three real-world software projects. 
Both sequential and parallel version of a conventional single objective evolutionary algorithm are implemented. 
The sequential version  is based on common programming approach using C++, and the parallel version is based on GPGPU programming approach using CUDA.
%The objective is to minimize the overall duration a software project, while satisfying the dependencies between work packages and constraints of resources allocation in the software project.
Results indicate that even a relatively cheap graphic card (GeForce GTX 970) can speed up the optimization process significantly.
We believe that deploy parallel evolutionary algorithm based on GPU  may fit many applications for other software project management problems, since software projects often have complex inter-related work packages and resources, and are typically characterized by large scale problems which optimization process ought to be accelerated by parallelism.
\vspace{-2mm}
\keywords{Software project management, Evolutionary algorithm, NVidia CUDA, Parallel computing, GPGPU}
\vspace{-2mm}
\end{abstract}
