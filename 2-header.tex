\tableofcontents
%
\mainmatter              % start of the contributions
%
\title{Solving Project Management Problem with Paralleled Evolutionary Algorithm}
%
\titlerunning{Project Management}  % abbreviated title (for running head)
%                                     also used for the TOC unless
%                                     \toctitle is used
%
\author{Jian Ren\inst{1} \and Jinghui Hu\inst{2} \and Xu Wang}
%
\authorrunning{Jian Ren et al.} % abbreviated author list (for running head)
%
%%%% list of authors for the TOC (use if author list has to be modified)
\tocauthor{Jian Ren, Jinghui Hu, Xu Wang}
%
\institute{Beihang University, Beijing 100191, China\\
\email{\{renjian, hujinghui, bhwangxu\}@buaa.edu.cn}
}

\maketitle

\begin{abstract}
In this paper, we focus on software project managers’ needs for
software project planning. Firstly, we briefly introduce
the background and current state of Software Project Management Problem (SPMP).
The software project management problem mainly includes resources allocation
and work packages scheduling. Our goal is to minimize the overall duration of a software
project, while satisfying the dependencies between work packages and constraints of resources
allocation in the software project. Finding an optimal solution for above-mentioned software
project problem is NP-hard. We learn from search based software engineering approach to
analyze and solve software project management problem. We implement both sequential and
parallel version applications, which are aim to solve the software project management
problem. The sequential version application is based on common programming approach using
C++ programming language, and the parallel version application is based on GPGPU programming
approach using CUDA C++ API. We redesign search based evolutionary algorithm to cater for our
purpose of parallel programming on GPU. Finally, we conduct a comparison experiment to verify
the parallel evolutionary algorithm does improve computational efficiency and evolutionary
algorithm always converge to nearly optimal solutions. 
\keywords{Software project management, Evolutionary algorithm, Paralleled Optimization Problem}
\end{abstract}
